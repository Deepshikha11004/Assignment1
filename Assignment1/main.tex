\documentclass[journal, 12pt, twocolumn]{IEEEtran}
\usepackage{enumitem}
\usepackage{enumerate}
\usepackage{amsmath}
\usepackage{textcomp}
\usepackage{multirow}
\usepackage[utf8]{inputenc}
\usepackage[margin=0.75in]{geometry}


\begin{document}
\title{Assignment 1(ICSE 2018 BOARD)}
\author{Deepshikha(CS21BTECH11016)}
\maketitle

\textbf{5.a}
The 4th term of a G.P. is 16 and the 7th terms is 128. Find the first term and common ratio of the series.


\textbf{Generalised:}Let the first term of the G.P. be a and common ratio r
, a\textsubscript m = p and a\textsubscript n = q be the m th  and n th term of G.P. respectively.


Therefore,
\begin{equation}
    a_m=ar^{(m-1)} = p
\end{equation}
\begin{equation}
    a_n=ar^{(n-1)} = q
\end{equation}

Dividing the equation $(2)\div$(1),
\begin{align*}
    \frac{ar^{(n-1)}}{ar^{(m-1)}}=\frac{q}{p} 
\end{align*}


\[r^{(n-m)}=\frac{q}{p}\]



\begin{equation}
    r =\left\{\frac{q}{p}\right\}^{(\frac{1}{n-m})}
\end{equation}

Equation (3) gives the value of r.

Put equation (3) in equation (1) gives value of a ,
\begin{equation}
    a=\left\{\frac{p^{(n-1)}}{q^{(m-1)}}\right\}^{({\frac{1}{n-m}})}
\end{equation}

Thus, equation (3) and (4) gives value of r and a respectively.




\textbf{Solution:}
Substituting m=4,n=7,p=16,q=128 in equations(3) and (4),
we get,
\[r=8^{(\frac{1}{3})}\]
\[r=2\]
and,
\[ a=\left\{\frac{16^{6}}{128^{3}}\right\}^{({\frac{1}{3}})}\]
\[a=2\]

\begin{table}[ht]
    \centering
    \begin{tabular}{|c|c|c|c|c|c|}
    \hline
    Variable & Value & Description\\
    \hline\hline
    $a_4$ & 16 & Fourth term\\
    $a_7$ & 128 & Seventh term\\
    a & 2 & First term\\
    r & 2 & Common diff.\\
    \hline
    \end{tabular}
    \caption{GP}
    \label{Table}
\end{table}

Therefore, 

First term=2 and,


Common difference=2




\end{document}

